%%%%%%%%%%%%%%%%%%%%%%%%%%%%%%%%%%%%%%%%%%%%%%%%%%%%%%%%%%%%%%%%%%%%%%%%%%%%%%%%
%2345678901234567890123456789012345678901234567890123456789012345678901234567890
%        1         2         3         4         5         6         7         8

%\documentclass[letterpaper, 10 pt, conference]{ieeeconf}  % Comment this line out
                                                          % if you need a4paper
\documentclass[a4paper, 10pt, conference]{ieeeconf}      % Use this line for a4
                                                          % paper
                                                          \usepackage[ansinew]{inputenc}
\usepackage{eurosym}
\usepackage{amssymb}
\usepackage[ansinew]{inputenc}
\usepackage{amsmath}
\usepackage{latexsym}
\usepackage{epsfig} %(this is to import figures)
\usepackage{psfrag}
\usepackage{dsfont}
\usepackage{nomencl}

\newtheorem{assumption1}{Assumption}
\newtheorem{assumption2}[assumption1]{Assumption}


\newtheorem{remark1}{Remark}

\IEEEoverridecommandlockouts                              % This command is only
                                                          % needed if you want to
                                                          % use the \thanks command
\overrideIEEEmargins
% See the \addtolength command later in the file to balance the column lengths
% on the last page of the document



% The following packages can be found on http:\\www.ctan.org
%\usepackage{graphics} % for pdf, bitmapped graphics files
%\usepackage{epsfig} % for postscript graphics files
%\usepackage{mathptmx} % assumes new font selection scheme installed
%\usepackage{times} % assumes new font selection scheme installed
%\usepackage{amsmath} % assumes amsmath package installed
%\usepackage{amssymb}  % assumes amsmath package installed

\title{\LARGE \bf
A New Approach to Optimal Energy Management with Discrete Control}

%\title{\LARGE \bf
%A New Approach to Optimal Energy Demand Side Management with
%Discrete Control Variables}

%\author{ \parbox{3 in}{\centering Huibert Kwakernaak*
%         \thanks{*Use the $\backslash$thanks command to put information here}\\
%         Faculty of Electrical Engineering, Mathematics and Computer Science\\
%         University of Twente\\
%         7500 AE Enschede, The Netherlands\\
%         {\tt\small h.kwakernaak@autsubmit.com}}
%         \hspace*{ 0.5 in}
%         \parbox{3 in}{ \centering Pradeep Misra**
%         \thanks{**The footnote marks may be inserted manually}\\
%        Department of Electrical Engineering \\
%         Wright State University\\
%         Dayton, OH 45435, USA\\
%         {\tt\small pmisra@cs.wright.edu}}
%}

\author{P. Balaguer, J.C. Alfonso, Ignacio and Jose% <-this % stops a space
%\thanks{This work was not supported by any organization}% <-this % stops a space
\thanks{P. Balaguer and J.C. Alfonso are with Departament d'Enginyeria de Sistemes Industrials i
Disseny, Universitat Jaume I de Castell�, Castell�, Spain.
        {\tt\small pbalague@uji.es, calfonso@uji.es}}%
\thanks{J. Zhang and X. Xia are with the Centre of New Energy Systems, Department
of Electrical, Electronic and Computer Engineering, University of
Pretoria, South Africa.
        {\tt\small zhang@up.ac.za, xxia@up.ac.za}}
}

\begin{document}



\maketitle
\thispagestyle{empty}
\pagestyle{empty}


%%%%%%%%%%%%%%%%%%%%%%%%%%%%%%%%%%%%%%%%%%%%%%%%%%%%%%%%%%%%%%%%%%%%%%%%%%%%%%%%
\begin{abstract}

In this article, we present a new approach to the solution of
optimization problems with discrete control variables modeled by
binary integer programs (BIP). The solution of BIP is
computationally demanding when the number of the BIP variables
increase. Two instances that increase the number BIP variables in
practical applications are the reduction of the discretization
sampling time and the increase of the optimization time period. The
proposed approach transforms a single BIP optimization into a linear
program (LP) and $N$ feasibility BIP's, with less number of
variables. The reduction of the number of variables increases the
algorithm speed in providing a solution. The approach permits to
solve optimization problems with longer time intervals and with a
higher number of control variables, while being computationally
tractable. A case study on the electricity cost minimization in a
pumping station shows the applicability of the method.

\end{abstract}


%%%%%%%%%%%%%%%%%%%%%%%%%%%%%%%%%%%%%%%%%%%%%%%%%%%%%%%%%%%%%%%%%%%%%%%%%%%%%%%%
\section{INTRODUCTION}
The presence of discrete control variables transforms standard
continuous optimization problems~\cite{bertsimas:97} into integer
optimization problems, which are computationally more demanding and
shown to be NP-hard~\cite{williams:09}. Integer optimization
applications are prevalent in electric power systems such as the
minimization of electricity cost by means of loads
management~\cite{load:01}~\cite{xia:09}, optimization of hybrid
systems operation~\cite{mix:04}~\cite{vieira:09}, reactive power and
voltage control~\cite{reactive2:09}, and on optimal energy
management in liberalized markets~\cite{liberalized:03}, among
others.

The previously introduced minimization problems are normally cast on
time periods of 24 hours because of the periodic nature of the
problem to be solved. The common approach to the solution of integer
minimization problem is to discretize the continuous objective
function and constraints using a one hour discretization sampling
time~\cite{load:01}~\cite{xia:09}~\cite{mix:04}~\cite{vieira:09}~\cite{reactive2:09}.

Although useful practical results may be accomplished using the
discretization process, the procedure has fundamental limitations
that prevent their utilization as a general approach. The main
problem relies on the curse of dimensionality problem of integer
programs~\cite{williams:09}. From a practical point of view, the
curse of dimensionality imposes a trade-off between the
discretization time, equivalently the number of program variables,
and the optimality of the achieved solution. In fact optimality of
the solution is compromised by the discretization~\cite{sager:09}
because not only the minimum on-off time is determined by the
sampling time but also the on-off time instants are fixed by the
discretization procedure. In general the chosen discretization
pattern is unable to provide the optimal solution.

Moreover, the discretization approach also shows fundamental
limitations in other settings, for instance, when incorporating
predictions on the optimization~\cite{vieira:09}. In this case the
problem time period can be enlarged beyond the 24 hours if longer
forecast periods are available, thus increasing the number of the
program variables. On the other hand, time period of 24 hours may
not be large enough in certain applications. Even if a feasible
solution to the minimization problem is found, it has been shown
that the feasible solution, when periodically repeated, may become
infeasible~\cite{xia:11}. Prediction time periods and feasibility
issues of repeated optimal solutions motivate the necessity of
optimization procedures that can enlarge the optimization time
horizon.

%Other approaches to the solution of integer programs include
%relaxation procedures~\cite{reactive1_02}, where suboptimal
%solutions are found by rounding solutions to their nearest integer
%value, genetic algorithms~\cite{genetic:05}, and swarm
%optimization~\cite{swarm:09}. Genetic algorithms and swarm
%optimization techniques also find suboptimal solutions and require a
%fine tuning of algorithm parameters, although they increase the
%class of problems to be solved.

In this article we present a new approach for solving binary integer
programs (BIP) that increases the time horizon without compromising
problem solvability due to computational burden. In this way the
solution is given in a two steps procedure. In the first step, the
BIP is transformed into a linear program (LP) by integration of the
binary variables. The solution of the LP provides the optimal energy
consumption integral per interval. In the second step, the optimal
energy value calculated is supplied in a binary way, by solving a
BIP feasibility problem. In this way, a unique BIP is transformed
into a LP plus $N$ feasibility BIP's related to shorter time
intervals with fewer variables. The reduction of the number of
variables presents an advantage because algorithms work faster with
fewer variables. Moreover, in the second stage, the algorithm only
requires a feasible solution, not an optimal one. Finally, the
approach permits to keep the complexity of the feasible BIP bounded
while the optimization period can be made arbitrarily large,
increasing its practical utility.

\section{PROBLEM FORMULATION}



\subsection{Problem Statement}
We consider the optimal energy cost management of a number of $M$
loads during a time period $T$ by minimizing the following cost
function

\begin{equation}\label{eq:min_ps}
J=\sum_{m=1}^{M}\int_{0}^{T}C(t)P_m(t)dt
\end{equation}

\noindent where $C(t)$ is the energy cost, and $P_m(t)$ is the power
consumption of load $m$.

\begin{assumption1}\label{ass:ss}
Function $C(t)$ is piecewise constant with $N$ intervals per period.
\end{assumption1}
%Normally the electricity price tariff is given by piecewise
%functions.

The power consumption is controlled by discrete variables $u_m(t), \
\ m=1,\ldots,M$, where

\begin{equation*}
u_m(t) = \left\{
\begin{array}{rl}
1, & \text{load m is on}\\
0, & \text{load m is off}
\end{array} \right.
\end{equation*}

\begin{assumption2}\label{ass:piecewise}
The power consumption of load $m$, $P_m(t)$ is given by
$P_m(t)=P_{m}^{ss}u_m(t)$, where $P_{m}^{ss}$ is the steady state
power consumption, and $u_m(t)$ is the discrete control action.
\end{assumption2}

The objective function~(\ref{eq:min_ps}) is to be minimized subject
to the following constraints

\begin{eqnarray}
s.t.&&\dot{x}(t)=Ax(t)+B_ww(t)+Bu(t) \label{eq:ss}\\
&&x_{min} \leq x(t) \leq x_{max}\label{eq:bounds}\\
&&u(t) \in \{0,1\}\label{eq:discrete}
\end{eqnarray}


\noindent where equation~(\ref{eq:ss}) is a continuous time dynamic
model represented in state space form~\cite{ogata:96} with state
vector $x(t) \in \mathbb{R}^{n}$, disturbance vector $w(t) \in
\mathbb{R}^{D}$, input vector $u(t) \in \mathbb{R}^{M}$, and dynamic
matrix $A \in \mathbb{R}^{n \times n}$, disturbance matrix $B_w \in
\mathbb{R}^{n \times D}$ and input matrix $B \in \mathbb{R}^{n\times
M}$. Equation~(\ref{eq:bounds}) are bounds on the state $x(t)$ and
equation~(\ref{eq:discrete}) refers to the discrete value of control
actions.

The BIP mathematical program set by cost function~(\ref{eq:min_ps})
and constrains~(\ref{eq:ss})-(\ref{eq:discrete}) captures the energy
cost minimization of a continuous time dynamical system with bounds
on the state values. The mathematical formulation captures
significant practical problems such as the energy management of a
colliery~\cite{xia:09}, peak load management in steel
plants~\cite{ashok:06}, demand side management of a mine winder
system~\cite{xia_mine:11}, optimization of wind-hydro plant
operation~\cite{vieira:09}, and energy cost minimization on pumping
stations~\cite{moradi:08}, among others. Moreover further
constraints commonly used in industrial settings such production
constrains, storage constraints, sequential constraints, and down
time of machines, among others can be cast in the framework
proposed~\cite{load:01}.
\section{PROBLEM TRANSFORMATION}
The complexity of solving the BIP is circumvented by a two step
approach. First the BIP is transformed into a LP that can be solved
more easily. Secondly $N$ simpler feasibility BIP are solved to
recover the discrete control actions.

\subsection{From BIP to LP}
The minimization of cost function~(\ref{eq:min_ps}), subject
to~(\ref{eq:ss})-(\ref{eq:discrete}), is a linear BIP because the
on-off nature of the control actions. However if the binary decision
variable $u_m(t)$ is  integrated, the integral $\int_{\Delta
t}u_m(t)dt$ is no longer binary but real, thus yielding a linear
program. Following assumptions~\ref{ass:ss} and~\ref{ass:piecewise},
the cost function~(\ref{eq:min_ps}) can be rewritten as

\begin{equation}\label{eq:Jdisc}
J=\sum_{m=1}^{M}\sum_{i=1}^{N}C_{i}P_{m}^{ss}\int_{t_{i-1}}^{t_i}u_m(t)dt
\end{equation}

\noindent where $U_{mi}=\int_{t_{i-1}}^{t_i}u_m(t)dt$ is the
operation time of load $m$ scheduled during time period $\Delta
t_i=t_{i}-t_{i-1}$, at constant price $C_{i}$. Figure~\ref{fig:var}
presents the time interval $T$ divided into intervals $\Delta t_i$
with constant electricity price, together with the variables
associated to each interval.
%In this way, for each load, we have transformed a binary variable
%$u_m(t)$ in a real variable $U_{mi}=\int_{\Delta t_i}u_m(t)dt$.
The objective function~(\ref{eq:Jdisc}) is rewritten as

\begin{equation}\label{eq:LPJ}
J=\sum_{m=1}^{M}P_{m}^{ss}\sum_{i=1}^{N}C_{i}U_{mi}
\end{equation}

\noindent which is a linear function with $MN$ decision variables
$U_{mi}$.

%Remark: The cost function is discretizated

\begin{figure}[thpb]
    \centering
    \psfrag{0}[][]{\fontsize{8}{8}\selectfont\textsf{$0$}}
    \psfrag{1}[][]{\fontsize{8}{8}\selectfont\textsf{$C_1$}}
    \psfrag{2}[][]{\fontsize{8}{8}\selectfont\textsf{}}
    \psfrag{im1}[][]{\fontsize{8}{8}\selectfont\textsf{$i-1$}}
    \psfrag{i}[][]{\fontsize{8}{8}\selectfont\textsf{$C_i$}}
    \psfrag{Nm1}[][]{\fontsize{8}{8}\selectfont\textsf{}}
    \psfrag{N}[][]{\fontsize{8}{8}\selectfont\textsf{$C_N$}}

    \psfrag{t0}[][]{\fontsize{8}{8}\selectfont\textsf{$t_0$}}
    \psfrag{t1}[][]{\fontsize{8}{8}\selectfont\textsf{$t_1$}}
    \psfrag{tim1}[][]{\fontsize{8}{8}\selectfont\textsf{$t_{i-1}$}}
    \psfrag{ti}[][]{\fontsize{8}{8}\selectfont\textsf{$t_i$}}
    \psfrag{tNm1}[][]{\fontsize{8}{8}\selectfont\textsf{$t_{N-1}$}}
    \psfrag{tN}[][]{\fontsize{8}{8}\selectfont\textsf{$t_N$}}

    \psfrag{U1}[][]{\fontsize{8}{8}\selectfont\textsf{$U_{1}$}}
    \psfrag{U2}[][]{\fontsize{8}{8}\selectfont\textsf{}}
    \psfrag{Uim1}[][]{\fontsize{8}{8}\selectfont\textsf{$U_{(i-1)}$}}
    \psfrag{Ui}[][]{\fontsize{8}{8}\selectfont\textsf{$U_{i}$}}
    \psfrag{UNm1}[][]{\fontsize{8}{8}\selectfont\textsf{}}
    \psfrag{UN}[][]{\fontsize{8}{8}\selectfont\textsf{$U_{N}$}}

    \psfrag{V0}[][]{\fontsize{8}{8}\selectfont\textsf{$x(t_{0})$}}
    \psfrag{V1}[][]{\fontsize{8}{8}\selectfont\textsf{$x(t_{1})$}}
    \psfrag{Vim1}[][]{\fontsize{8}{8}\selectfont\textsf{$x(t_{i-1})$}}
    \psfrag{Vi}[][]{\fontsize{8}{8}\selectfont\textsf{$x(t_{i})$}}
    \psfrag{VNm1}[][]{\fontsize{8}{8}\selectfont\textsf{$x(t_{N-1})$}}
    \psfrag{VN}[][]{\fontsize{8}{8}\selectfont\textsf{$x(t_{N})$}}

    \psfrag{Deltat}[][]{\fontsize{8}{8}\selectfont\textsf{$\Delta t_{i}$}}

%    \psfrag{V0}[][]{\fontsize{8}{8}\selectfont\textsf{$V(0)$}}
%    \psfrag{U1}[][]{\fontsize{8}{8}\selectfont\textsf{$U(t_1)$}}
    %\psfrag{ep}[][]{\fontsize{8}{8}\selectfont\textsf{$EP$}}
%    \psfrag{t}[][]{\fontsize{8}{8}\selectfont\textsf{$T$}}
%    \psfrag{p}[][]{\fontsize{8}{8}\selectfont\textsf{$G(s)$}}
%    \psfrag{k}[][]{\fontsize{8}{8}\selectfont\textsf{$K^{-1}$}}
%    \psfrag{r}[][]{\fontsize{8}{8}\selectfont\textsf{$R$}}


    \includegraphics[scale=0.6]{nomenclature.eps}
    \caption{Problem Variables.}
    \label{fig:var}
\end{figure}

The constraints~(\ref{eq:ss})-(\ref{eq:discrete}) are also
integrated. First integrating the system dynamics~(\ref{eq:ss}) we
have

\begin{eqnarray}\label{eq:LPx}
x(t_i)&-&x(t_{i-1})=A\int_{t_{i-1}}^{t_{i}}x(t)dt+ \nonumber \\
&+&B_w\int_{t_{i-1}}^{t_{i}}w(t)dt+B\int_{t_{i-1}}^{t_{i}}u(t)dt.
\end{eqnarray}

\noindent Performing the change of variables
$X_{i}=\int_{t_{i-1}}^{t_{i}}x(t)dt$ and
$W_{i}=\int_{t_{i-1}}^{t_{i}}w(t)dt$, and reordering
expression~(\ref{eq:LPx}), the state value at time instant $t_{i}$
is

\begin{equation}\label{eq:vint}
x(t_i)=x(t_{i-1})+AX_{i}+B_wW_{i}+BU_{i}.
\end{equation}

\noindent The disturbance $W_{i}$ is known, whereas $X_{i}$ and
$U_{i}$ are the decision variables, hence by the variables grouping

\begin{eqnarray}
\bar{A}&=&
\begin{bmatrix}
A & B
\end{bmatrix}\\
\bar{V}_{i}&=&
\begin{bmatrix}
X_{i}\\
U_{i}
\end{bmatrix}
\end{eqnarray}


\noindent we have that, finally, equation~(\ref{eq:vint}) can be
expressed as

\begin{equation}\label{eq:ssdisc}
x(t_i)=x(t_{i-1})+B_wW_{i}+\bar{A}\bar{V}_{i}
\end{equation}

\noindent for $i=1,\ldots,N$.

Applying the state constraint~(\ref{eq:bounds}) to
equation~(\ref{eq:ssdisc}) yields

\begin{eqnarray}
x_{min}-x(0)-B_w\sum_{i=1}^{j}W_i&\leq& \bar{A}\sum_{i=1}^{j}\bar{V}_{i}\\
x_{max}-x(0)-B_w\sum_{i=1}^{j}W_i&\geq&
\bar{A}\sum_{i=1}^{j}\bar{V}_{i}
\end{eqnarray}

\noindent for $j \in [1,\ldots,N]$, which can be finally written as

\begin{equation}\label{eq:axb}
\begin{bmatrix}
-A_{T}\\
A_{T}
\end{bmatrix}
\begin{bmatrix}
\bar{V}_{1}\\
\vdots\\
\bar{V}_{N}\\
\end{bmatrix}
\leq
\begin{bmatrix}
-\mathds{1}_{1\times n}\otimes(x_{min}-x_0)+B_w\otimes W\\
\mathds{1}_{1\times n}\otimes(x_{max}-x_0)-B_w\otimes W
\end{bmatrix}
\end{equation}


\noindent where $\otimes$ is the matrix Kronecker product~\cite[p.
399]{matrix:09}, and

\begin{eqnarray}\label{eq:axbb}
A_{T}=
\begin{bmatrix}
\bar{A} & 0 & \ldots & 0\\
\bar{A} & \bar{A} & \ldots & 0\\
\bar{A} & \bar{A} & \ldots & \bar{A}
\end{bmatrix}, \ \
W=
\begin{bmatrix}
W_{1}\\
W_{1}+W_{2}\\
\vdots\\
\sum_{i}^{N}W_{i}\\
\end{bmatrix}
\end{eqnarray}


\noindent In this way the inequality~(\ref{eq:axb}) is written as
the standard inequality $Ax\leq b$ of linear programs.



The integral of the control action is also limited by

\begin{equation}\label{eq:cca}
0_{M\times 1} \leq U_{i} \leq \mathds{1}_{M\times 1} \Delta t_{i}
\end{equation}

%\begin{equation}\label{eq:cca}
%0\leq U_{i} \leq \Delta t_{i}
%\end{equation}


\noindent which sets the constraints on the control action. In fact
constraint~(\ref{eq:cca}) sets that the minimum value of the control
action is 0, that is the loadis off during all the time interval. On
the other hand, the bound $\Delta t_{i}$, sets that the load can be,
at most, on during all the time interval.

%\begin{remark1}
%The variables $X_{i}$ do not appear on the cost function but only on
%the constraints because the minimization problem is an energy cost
%minimization. However, it is straightforward to include $X_{i}$ in
%the cost function when the state integral is also to be minimized.
%In this case the following constraint on the state integral should
%also be added to the LP
%
%\begin{equation}
%x_{min}\Delta t_{i}\leq X_{i} \leq x_{max}\Delta t_{i}
%\end{equation}
%
%\end{remark1}







\subsection{Discrete Control Actions}
Once the LP defined by cost function~(\ref{eq:LPJ}) and
constraints~(\ref{eq:axb})-(\ref{eq:cca}) is solved, the optimal
energy consumption per period $U_{i}, \ \ i=1,\ldots, N$ is known.
However, the energy is supplied by a discrete control variable
$u(t)$, but the previos stage only provides its integral value, that
is, $\int_{\Delta t}u(t)dt$. The discrete control variable is
recovered by finding a feasible BIP solution for each period
$i=1,\ldots,N$ with constant electricity price. In this way, period
$i$ with constant electricity price is discretized in $K$ parts and
a solution is given by any feasible solution of the BIP with
constraints

\begin{eqnarray}
\frac{\Delta t_i}{K}\sum_{k=0}^{K-1}u_m(k)=U_{mi}, \ \ m=1,\ldots,M \label{eq:bipener}\\
x(k+1)=A_dx(k)+B_{wd}w(k)+B_du(k)\label{eq:bipss}\\
x_{min} \leq x(k) \leq x_{max}\label{eq:bipbounds}\\
u(k) \in \{0,1\}\label{eq:bipdiscrete}
\end{eqnarray}

%\begin{eqnarray}
%&&\frac{\Delta t_i}{K}\sum_{k=1}^{K}u_m(k)=U_{mi}, \ \ m=1,\ldots,M \label{eq:bipener}\\
%&&x(k+1)=A_dx(k)+B_{wd}w(k)+B_du(k)\label{eq:bipss}\\
%&&x_{min} \leq x(k) \leq x_{max}\label{eq:bipbounds}\\
%&&u(k) \in \{0,1\}\label{eq:bipdiscrete}
%\end{eqnarray}

\noindent where~(\ref{eq:bipener}) constraints the supplied energy
to be equal to the optimal value obtained by the previous LP,
equations~(\ref{eq:bipss}) and~(\ref{eq:bipbounds}) are the discrete
time version of the continuous system dynamics,
and~(\ref{eq:bipdiscrete}) recovers the discrete nature of the
control action. Matrices $A_d \in \mathbb{R}^{n\times n}$, $B_{wd}
\in \mathbb{R}^{n\times D}$ and $B_d \in \mathbb{R}^{n\times M}$ are
obtained by zero order hold discretization from their continuous
time counterparts. Discrete time vectors have the same dimensions as
their continuous counterparts, hence $x(k) \in \mathbb{R}^{n}$,
$w(k) \in \mathbb{R}^{D}$, and $u(k) \in \mathbb{R}^{M}$.

\begin{remark1}\label{rem:zoh}
Dynamic model~(\ref{eq:bipss}) is the zero order hold discretization
of continuous time model~(\ref{eq:ss}). Moreover, during sampling
times, control action is constant due to its integer nature, as a
result both models~(\ref{eq:bipss}) and~(\ref{eq:ss}) yield the same
value at sampling times~\cite{ogata:95}, thus the discretization is
exact.
\end{remark1}

The BIP defined by~(\ref{eq:bipener})-(\ref{eq:bipdiscrete}) can be
written in standard form as follows. First, the
constraint~(\ref{eq:bipener}) is written in matrix form as

\begin{equation}\label{eq:equalbip}
\mathds{1}_{M\times K}
\begin{bmatrix}
u(0)\\
u(1)\\
\vdots\\
u(K-1)\\
\end{bmatrix}=U_{i}\frac{K}{\Delta t_i}.
\end{equation}

Next, equation~(\ref{eq:bipss}) is written, for all $k=0,\ldots,K$,
in matrix form as
%equation~(\ref{eq:dsst}). Now



\begin{eqnarray}
\begin{bmatrix}
x(0)\\
x(1)\\
\vdots\\
x(K)
\end{bmatrix}&=&
\begin{bmatrix}
I_{n\times n}\\
A_d^1\\
\vdots\\
A_d^{K}\\
\end{bmatrix}\otimes x(0)+
\bar{B}_{wd}
\begin{bmatrix}
w(0)\\
w(1)\\
\vdots\\
w(K-1)
\end{bmatrix} \nonumber\\
&+& \bar{B}_d
\begin{bmatrix}
u(0)\\
u(1)\\
\vdots\\
u(K-1)
\end{bmatrix}\label{eq:dsst}
\end{eqnarray}

\noindent with

\begin{eqnarray}
\bar{B}_{wd}&=&\begin{bmatrix}\label{eq:barbwdbd}
0_{n\times D} & 0_{n\times D} & \ldots & 0_{n\times D} \\
B_{wd} & 0_{n\times D}& \ldots & 0_{n\times D}  \\
A_dB_{wd} & B_{wd} & \ldots   & 0_{n\times D} \\
\vdots\\
A_d^{K-1}B_{wd} & A_d^{K-2}B_{wd} & \ldots   & B_{wd} \\
\end{bmatrix}
\end{eqnarray}
\begin{eqnarray}
\bar{B}_d&=&\begin{bmatrix}\label{eq:barbd}
0_{n\times M} & 0_{n\times M} & \ldots & 0_{n\times M} \\
B_d & 0_{n\times M}& \ldots & 0_{n\times M}  \\
A_dB_d & B_d & \ldots   & 0_{n\times M} \\
\vdots\\
A_d^{K-1}B_d & A_d^{K-2}B_d & \ldots   & B_d \\
\end{bmatrix}
\end{eqnarray}

Constraint~(\ref{eq:bipbounds}) applied to equation~(\ref{eq:dsst})
yields


\begin{eqnarray}\label{eq:axbbip}
&&\begin{bmatrix}
-\bar{B}_{d}\\
\bar{B}_{d}
\end{bmatrix}
\begin{bmatrix}
u(0)\\
u(1)\\
\vdots\\
u(K-1)\\
\end{bmatrix}
\leq \nonumber\\
&&
\begin{bmatrix} -x_{min}\otimes\mathds{1}_{K\times 1}+
\begin{bmatrix}
I_{n\times n}\\
A_d^1\\
\vdots\\
A_d^{K}\\
\end{bmatrix}\otimes x(0)+
\bar{B}_{wd}
\begin{bmatrix}
w(0)\\
w(1)\\
\vdots\\
w(K-1)\\
\end{bmatrix}\\
x_{max}\otimes\mathds{1}_{K\times 1}-
\begin{bmatrix}
I_{n\times n}\\
A_d^1\\
\vdots\\
A_d^{K-1}\\
\end{bmatrix}\otimes x(0)-
\bar{B}_{wd}
\begin{bmatrix}
w(0)\\
w(1)\\
\vdots\\
w(K-1)\\
\end{bmatrix}
\end{bmatrix}\nonumber\\
\end{eqnarray}

\noindent yielding the standard BIP linear inequality $Ax\leq b$.

\subsection{Sampling Time Selection}
The sampling time $T_s=\Delta t_{i}/K$ to obtain the BIP must be
chosen with care because, on the one hand, the smaller the sampling
time the larger the number of BIP variables, and, on the other hand,
if the sampling time is too large the resulting BIP problem may be
infeasible. Moreover, the sampling time selection may be used to
meet the integer constraints on equation~(\ref{eq:bipener}). Recall
that the term $\sum_{k}u_m(k)$ is an integer for $m=1 \ldots M$.
From equation~(\ref{eq:bipener}), $U_{mi}/T_s$ for $m=1 \ldots M$
must also be an integer to have a meaningful BIP. We assume that
$U_{mi} \in \mathbb{N}$, by rounding $U_{mi}$ up to the desired
accuracy degree. By choosing a sampling time equal to
$T_s=U_{gcd}/L$, with $U_{gcd}$ the greatest common divisor of
$\{U_{1i}, U_{2i},\ldots,U_{Mi}\}$, and $L \in \mathbb{N}$, we may
guarantee the integer value of $U_{mi}/T_s$ for $m=1 \ldots M$. It
is worth to mention that the term $L$, whenever $K=\Delta
t_{i}L/U_{gcd} \in \mathbb{N}$, does not have influence on the
optimality of the energy consumption but it has influence on the
feasibility and complexity of the BIP problem. In fact the BIP
number of time slots is given by $K=\Delta t_{i}L/U_{gcd}$. A lower
bound on the BIP complexity is provided by $L=1$ but the BIP may be
infeasible. In this case parameter $L$ must be increased to obtain a
feasible solution at the cost of increasing the BIP complexity
because the BIP variables number is given by $MK$, where $M$ is the
number of loads to be scheduled.



%One approach would be: First select a small $L$ value (i.e. a lower
%bound is $L=1$). If the problem is feasible then it is solved.
%Otherwise increase the variable $L$. The number of BIP variables is
%given by $MK$, where $M$ is the number of loads to be scheduled and
%$K$ is the number of time slots equal to $K=\Delta t_{i}L/U_{mcm}$,
%with $\Delta t_{i}$ the time interval length and $U_{gcd}$ the great
%common denominator. This has been modified in the article as
%follows:(new section)
%
%
%In fact, choosing $L=U_{gcd}$ we have that
%$U_{mi}/T_s=U_{mi}L/U_{gcd}=U_{mi} \in \mathbb{N}$. However, it is
%possible that for $L<U_{gcd}$ we also have that $U_{mi}L/U_{gcd} \in
%\mathbb{N}$. For $L=1$ the sampling time is equal to $U_{gcd}$.
%Increasing $L$ provides more freedom in the way energy is deployed
%on the period, although the number of the feasibility BIP variables
%increases. It is worth to mention that the term $L$ does not have
%influence on the optimality of the energy consumption but it has
%influence on the feasibility of the BIP problem.

%%%%%%%%%%%%%%%%%%%%

%The the complexity of finding a feasible solution to the BIP is
%determined by the discretization of the period with constant
%electricity price, because the discretization determines the number
%of the BIP variables. The sampling time is defined as $T_s=\Delta
%t_{i}/K$.
%
%The sampling time $T_s$ must be chosen with care, because integer
%constraints must be met. Recall that the term $\sum_{k}u_m(k)$ is an
%integer for $m=1 \ldots M$. From equation~(\ref{eq:bipener}),
%$\sum_{k}u_m(k)=U_{mi}/T_s$ for $m=1 \ldots M$, thus to have a
%meaningful BIP $U_{mi}/T_s$ for $m=1 \ldots M$, must be an integer.
%
%By choosing a sampling time equal to $T_s=U_{mcm}/L$, with $U_{mcm}$
%the minimum common multiple of $\{U_{1i}, U_{2i},\ldots,U_{Mi}$, and
%$L \in \mathbb{N}$, we can guarantee the integer value of $U_mi/T_s$
%for $m=1 \ldots M$. For $L=1$ the sampling time is equal to $U_mcm$.
%Increasing $L$ provides more freedom in the way energy is deployed
%on the period, although the number of the feasibility BIP variables
%increases. It is worth to mention that the term $L$ does not have
%influence on the optimality of the energy consumption but it has
%influence on the feasibility of the BIP problem.
%
%The number of subintervals $K \in \mathbb{Z}$ is related to $L$ by
%$K=\Delta t_{i}L/U_{mcm}$ when the proposed sampling time is used.
%In order to exactly discretize the given interval it should be
%accomplished that $\Delta t_{i}/U_{mcm} \in \mathbb{Z}$ what, in
%general, does not hold. However, by increasing the value $L$ the
%term $K=\Delta t_{i}L/U_{mcm}$ approximates an integer. As a result,
%by increasing the value of $L$ we assure that the BIP is feasible
%and that the time discretization error is diminished but we increase
%the number of variables of the feasible BIP.



%\section{PROBLEM SOLUTION}
%
%A summary of the solution methodology proposed is divided into 5
%steps as follows:
%
%\begin{itemize}
%\item[Step 1:]Define $N$ as the number of time intervals with
%constant electricity price and construct the objective
%function~(\ref{eq:LPJ}) and constraint~(\ref{eq:axb}) using $A_{T}$
%and $W$ as in~(\ref{eq:axbb}).
%\item[Step 2:] Solve the previous LP~(\ref{eq:LPJ}),~(\ref{eq:axb}) and obtain $U_i$ for $i=1,\ldots,N$. $U_i$ provides the optimal energy consumption per period
%$i=1,\ldots,N$ with constant electricity price.
%\item[Step 3:] Select $L=1$ and construct the
%BIP~(\ref{eq:bipener})-(\ref{eq:bipdiscrete}) by
%constructing~(\ref{eq:equalbip}), $\bar{B}_{wd}$, and $\bar{B}_{d}$,
%as defined in~(\ref{eq:barbwdbd}) and~(\ref{eq:barbd}),
%respectively. Repeat the procedure for each time interval
%$i=1,\ldots,N$ with $U_i\neq 0$.
%\item[Step 4:] Find a feasible solution of the
%BIP~(\ref{eq:bipener})-(\ref{eq:bipdiscrete}) and apply the feasible
%solution $u_m(k)$ for each load $m=1,\ldots,M$, and for each
%$k=1,\ldots,K$.
%\item[Step 5:] In case that the BIP~(\ref{eq:bipener})-(\ref{eq:bipdiscrete}) is infeasible, repeat from
%step 3 with $L=L+1$.
%\end{itemize}


\section{CASE STUDY}
Water supply systems present high-energy consumption values, which
corresponds to the major expenses of these systems. In this section
we consider the optimal operation of the water supply system
presented in Figure~\ref{fig:ps}. It consists on three reservoirs
and two pumps to be controlled. A constant flow input of $F_{e}=1/6$
Kilo-liter per minute, or Kl/min, is entering reservoir 1
(Fig.~\ref{fig:ps}), while reservoirs 2 and 3 must supply a constant
flow of $F_{o2}=F_{o2}=1/12$ Kl/min each. The maximum volume of
reservoir 1, 2, and 3 is $0.4$ Ml, $0.25$ Ml and $0.25$ Ml,
respectively whereas the minimum volume is $0.02$ Ml, $0.02$ Ml, and
$0.02$ Ml. Pump 1 provides a nominal flow of $F_1=1/2$ Kl/min and a
power consumption of $P_1=5$ KW. Pump 2 provides a nominal flow of
$F_2=3/5$ Kl/min and the power consumption is $P_2=6$ KW because not
only the flow is higher but also the piezometric height of reservoir
3 is higher than reservoir 2. The objective is to minimize the
operation electricity cost while maintaining the reservoirs volume
constraints.

%Electricity cost is a function of the energy consumption and the
%daily electrical tariff.

The electricity price, in cents of euro per KWh or c \euro/KWh, has
6 intervals during a 24 hours period, and is given by

\begin{eqnarray}
11.87\ \ c \text{\euro}/KWh, \ \ t&=&[0,6]\cup[22,24] \nonumber \\
14.11\ \ c \text{\euro}/KWh, \ \ t&=&[6,7]\cup[10,18] \nonumber \\
82.05\ \ c \text{\euro}/KWh, \ \ t&=&[7,10]\cup[18,22] \nonumber
\end{eqnarray}


that shows 6 time intervals of very different time duration, ranging
from 1 to 8 hours. Moreover it presents a high cost on the peak
hours.



\begin{figure}[thpb]
    \centering
    \psfrag{fi}[][]{\fontsize{8}{8}\selectfont\textsf{$F_e=1/6$ Kl/min}}
    \psfrag{fo2}[][]{\fontsize{8}{8}\selectfont\textsf{$F_{o2}=1/12$ Kl/min}}
    \psfrag{fo3}[][]{\fontsize{8}{8}\selectfont\textsf{$F_{o3}=1/12$ Kl/min}}
    \psfrag{q1}[][]{\fontsize{8}{8}\selectfont\textsf{$F_1=1/2$ Kl/min}}
    \psfrag{q2}[][]{\fontsize{8}{8}\selectfont\textsf{$F_2=3/5$ Kl/min}}
    \psfrag{reservoir1}[][]{\fontsize{8}{8}\selectfont\textsf{Reservoir 1}}
    \psfrag{reservoir2}[][]{\fontsize{8}{8}\selectfont\textsf{Reservoir 2}}
    \psfrag{reservoir3}[][]{\fontsize{8}{8}\selectfont\textsf{Reservoir 3}}
    \psfrag{pump1}[][]{\fontsize{8}{8}\selectfont\textsf{$P_1=5$ KW}}
    \psfrag{pump2}[][]{\fontsize{8}{8}\selectfont\textsf{$P_2=6$ KW}}


    \includegraphics[scale=0.7]{pump_scheme.eps}
    \caption{Water management pumping scheme.}
    \label{fig:ps}
\end{figure}




\subsection{Solution of the LP}

The LP cost function is
\begin{equation}
J=P_{1}\sum_{i=1}^{6}C_{i}U_{1i}+P_{2}\sum_{i=1}^{6}C_{i}U_{2i}
\end{equation}

%with $12$ decision variables $U_{1i},\ \ U_{2i} , \ \ i=1,\ldots,6$.
The constraints~(\ref{eq:axb})-(\ref{eq:cca}) are defined by


\begin{eqnarray}
A=
\begin{bmatrix}
0 & 0 & 0\\
0 & 0 & 0\\
0 & 0 & 0
\end{bmatrix}, \ \
B=
\begin{bmatrix}
-F_{1} & -F_{2}\\
F_{1} & 0\\
0 & F_{2}\\
\end{bmatrix}, \ \
W_{i}=
\begin{bmatrix}
F_{e}\Delta t_{i}\\
F_{o2}\Delta t_{i}\\
F_{o3}\Delta t_{i}
\end{bmatrix}
\end{eqnarray}

%\begin{eqnarray}
%W_{i}=
%\begin{bmatrix}
%F_{e}\Delta t_{i}\\
%F_{o2}\Delta t_{i}\\
%F_{o3}\Delta t_{i}
%\end{bmatrix}
%\end{eqnarray}


\noindent and the decision variables are

\begin{eqnarray}
%X_{i}=
%\begin{bmatrix}
%X_{1i}\\
%X_{2i}\\
%X_{3i}\\
%\end{bmatrix}, \ \
U_{i}=
\begin{bmatrix}
U_{1i}\\
U_{2i}\\
\end{bmatrix},
\ \ i=1,\ldots,6
\end{eqnarray}

%for $i=1,\ldots,6$.

Two distinct optimization problems are considered

\begin{enumerate}
\item Case 1: a 24 hours scheduling with initial state given by $x(0)=[2\ \ 1\ \ 1]^{T}10^{-1}$ Ml.

%\item Case 2: a 24 hours scheduling with initial state given by $x(0)=[4\ \ 0.2\ \ 0.2]^{T}10^{-1}$
%Ml, which corresponds to reservoir 1 full and reservoirs 2 and 3
%empty.

%\item Case 3: a 48 hours scheduling with initial state given by $x(0)=[2\ \ 1\ \ 1]^{T}10^{-1}$ Ml.

\item Case 2: a 48 hours scheduling with initial state given by $x(0)=[2\ \ 1\ \ 1]^{T}10^{-1}$ Ml.
\end{enumerate}

Both, Case 1 and Case 2 have the same initial conditions. Case 2 is
included in order to show how optimization period can be enlarged
and still yield a solvable problem.
%The Case 3 has the same initial conditions as Case 1 and it is
%included in order to show how optimization period can be enlarged
%and still yield a solvable problem.
Tables~\ref{tbl:LP1} and ~\ref{tbl:LP3} presents the minimization
results for Case 1 and Case 2, respectively. Case 1 is a LP with 12
decision variables, whereas Case 2 requires the solution of a LP
with 24 decision variables. Both problems are solved in less than
0.5 seconds in a desktop PC (2.66 GHz, 2.49 GB RAM) using the
standard Matlab function ``linprog''.
%Tables~\ref{tbl:LP1}, \ref{tbl:LP2} and ~\ref{tbl:LP3} presents the
%minimization results for Case 1, Case 2, and Case 3, respectively.
%Control actions are given in minutes for each pump operation. The
%values have been rounded, hence we have a precision of half a
%minute.
\begin{table}[thpb]
\begin{center}\caption{Case 1 Optimal Energy per Period given by LP solution.}
\begin{tabular}{|c|c|c|c|c|c|c|c|}
\hline
Interval $i$  & 1 & 2 & 3 & 4 & 5 & 6 & Energy (min)\\
\hline
$U_{1i}$ (min) & 227 & 14 & 0 & 103 &  0 & 40 & 384\\
\hline
$U_{2i}$ (min) & 191 & 11 & 0 & 85 & 0 & 33 & 320\\
\hline
\end{tabular}
\label{tbl:LP1}
\end{center}
\end{table}
%
%\begin{table}[thpb]
%\begin{center}\caption{Case 2 Optimal Energy per Period given by LP solution.}
%\begin{tabular}{|c|c|c|c|c|c|c|c|}
%\hline
%Interval $i$  & 1 & 2 & 3 & 4 & 5 & 6 & Energy (min)\\
%\hline
%$U_{1i}$ (min) & 336 & 12 & 0 & 92 &  0 & 40 & 480\\
%\hline
%$U_{2i}$ (min) & 280 & 8 & 0 & 78 & 0 & 33 & 399\\
%\hline
%\end{tabular}
%\label{tbl:LP2}
%\end{center}
%\end{table}

\begin{table*}[thpb]
\begin{center}\caption{Case 2 Optimal Energy per Period given by LP solution.}
\begin{tabular}{|c|c|c|c|c|c|c|c|c|c|c|c|c|c|c|}
\hline
 Interval $i$  & 1 & 2 & 3 & 4 & 5 & 6 & 7 & 8 & 9 & 10 & 11 & 12 & Energy (min)\\
\hline
$U_{1i}$  (min) & 224 & 13 & 0 & 112 & 0 & 91 & 265 & 11 & 0 & 113 & 0 & 40 & 869\\
\hline
$U_{2i} $ (min) & 188 & 11 &  0 & 93 & 0 & 82 & 213 & 9 & 0 & 95 & 0 & 33 & 724\\
\hline
\end{tabular}
\label{tbl:LP3}
\end{center}
\end{table*}


\subsection{Solution of the BIP}
Tables~\ref{tbl:LP1} and ~\ref{tbl:LP3} provide the optimal energy
deployment per period with constant electricity price. In order to
obtain the discrete control action it is necessary to solve 4
feasibility BIP's in Case 1 and 8 feasibility BIP's in Case 2,
because in all solutions, during the time intervals with peak
electricity price both pumps are kept off.
%The feasible BIP is given by
%equations~(\ref{eq:axbbip})-(\ref{eq:equalbip}) that are completely
%defined by matrices $A_d$, $B_d$, and $D_d$, which are the
%zero-order hold discretization of matrices $A$, $B$, and $B_w$,
%respectively.
The sampling time used for discretization is show in the second row
of tables ~\ref{tbl:BIP1} and ~\ref{tbl:BIP3}, for the two cases
considered, whereas the first row is the selected $L$ value and the
last row shows the number of intervals $K$ used in each feasibility
BIP. Consider for instance Case 1. The largest $K$ is equal to 7,
which corresponds to the last time interval. The feasibility BIP has
14 variables, because there are 2 pumps. All cases are solved in
less than 1.5 seconds in a standard desktop PC using Matlab
functions ``linprog'' and ``bintprog''. Note that the commonly used
approach of direct discretization of cost function~(\ref{eq:min_ps})
subject to~(\ref{eq:ss})-(\ref{eq:discrete}) with one hour sampling
time yields an optimization BIP with 48 variables. The procedure was
implemented by means of the Matlab ``bintprog'' and stopped without
finding any solution because of ``maximum nodes reached without
converging''. Finally Figure~\ref{fig:pumps1} shows the pumps
control action together with the electricity price for Case 1. As
can be seen during the peak intervals both pumps are kept off.
Figure~\ref{fig:V1} shows the reservoir volume variations which are
always in accordance with reservoir capacity constraints. The
results for Case 2 can be seen in Figures~\ref{fig:pumps3}
and~\ref{fig:V3}.


%The reservoir volume variations are presented in Figure~\ref{fig:V1}
%and ~\ref{fig:V3}. The reservoir volume is always in accordance with
%reservoir capacity constraints.


\begin{table}[thpb]
\begin{center}\caption{Case 1 parameter $L$, sampling time $T_s$, and number of
intervals $K$ for each feasibility BIP. Zeros indicate that no
feasibility BIP is calculated because there is no energy to deploy
in the interval.} \label{tbl:BIP1}
\begin{tabular}{|c|c|c|c|c|c|c|}
\hline
BIP  & 1 & 2 & 3 & 4 & 5 & 6 \\
\hline
$L$ & 3 & 1 & 0 & 1 &  0 & 2\\
\hline
$T_s$ (min) & 60 & 12 & 0 & 80 & 0 & 17.14 \\
\hline
$K$  & 6 & 5 & 0 & 6 & 0 & 7 \\
\hline
\end{tabular}
\end{center}
\end{table}
%
%\begin{table}[thpb]
%\begin{center}\caption{Case 2 parameter $L$, sampling time $T_s$, and number of
%intervals $K$ for each feasibility BIP. Zeros indicate that no
%feasibility BIP is calculated because there is no energy to deploy
%in the interval.} \label{tbl:BIP2}
%\begin{tabular}{|c|c|c|c|c|c|c|}
%\hline
%BIP  & 1 & 2 & 3 & 4 & 5 & 6 \\
%\hline
%$L$ & 3 & 1 & 0 & 1 &  0 & 2\\
%\hline
%$T_s$ (min) & 90 & 7.5 & 0 & 80 & 0 & 17.14 \\
%\hline
%$K$  & 4 & 8 & 0 & 6 & 0 & 7 \\
%\hline
%\end{tabular}
%\end{center}
%\end{table}
%
\begin{table*}[thpb]
\begin{center}\caption{Case 2 parameter $L$, sampling time $T_s$, and number of
intervals $K$ for each feasibility BIP. Zeros indicate that no
feasibility BIP is calculated because there is no energy to deploy
in the interval.} \label{tbl:BIP3}
\begin{tabular}{|c|c|c|c|c|c|c|c|c|c|c|c|c|c|}
\hline
BIP  & 1 & 2 & 3 & 4 & 5 & 6 & 7 & 8 & 9 & 10 & 11 & 12 \\
\hline
$L$  & 3 & 1 & 0 & 1 & 0 & 2 & 3 & 1 & 0 & 1 & 0 & 2 \\
\hline
$T_s$ (min) & 60 & 12 &  0 & 96 & 0 & 40 & 72 & 8.57 & 0 & 96 & 0 & 17.14 \\
\hline
$K$  & 6 & 5 &  0 & 5 & 0 & 3 & 5 & 7 & 0 & 5 & 0 & 7 \\
\hline
\end{tabular}
\end{center}
\end{table*}


%Finally Figure~\ref{fig:pumps1} and~\ref{fig:pumps3} show the pumps
%control action together with the electricity price. As can be seen
%during the peak intervals both pumps are kept off. The reservoir
%volume variations are presented in Figure~\ref{fig:V1} and
%~\ref{fig:V3}. The reservoir volume is always in accordance with
%reservoir capacity constraints.


\begin{figure}[thpb]
    \centering
%    \psfrag{V0}[][]{\fontsize{8}{8}\selectfont\textsf{$V(0)$}}
%    \psfrag{U1}[][]{\fontsize{8}{8}\selectfont\textsf{$U(t_1)$}}
    %\psfrag{ep}[][]{\fontsize{8}{8}\selectfont\textsf{$EP$}}
%    \psfrag{t}[][]{\fontsize{8}{8}\selectfont\textsf{$T$}}
%    \psfrag{p}[][]{\fontsize{8}{8}\selectfont\textsf{$G(s)$}}
%    \psfrag{k}[][]{\fontsize{8}{8}\selectfont\textsf{$K^{-1}$}}
%    \psfrag{r}[][]{\fontsize{8}{8}\selectfont\textsf{$R$}}


    \includegraphics[scale=0.28]{pumps.eps}
    \caption{Case 1 pumps control actions.}
    \label{fig:pumps1}
\end{figure}

\begin{figure}[thpb]
    \centering
%    \psfrag{V0}[][]{\fontsize{8}{8}\selectfont\textsf{$V(0)$}}
%    \psfrag{U1}[][]{\fontsize{8}{8}\selectfont\textsf{$U(t_1)$}}
    %\psfrag{ep}[][]{\fontsize{8}{8}\selectfont\textsf{$EP$}}
%    \psfrag{t}[][]{\fontsize{8}{8}\selectfont\textsf{$T$}}
%    \psfrag{p}[][]{\fontsize{8}{8}\selectfont\textsf{$G(s)$}}
%    \psfrag{k}[][]{\fontsize{8}{8}\selectfont\textsf{$K^{-1}$}}
%    \psfrag{r}[][]{\fontsize{8}{8}\selectfont\textsf{$R$}}


    \includegraphics[scale=0.28]{V.eps}
    \caption{Case 1 volume variation and capacity constraints.}
    \label{fig:V1}
\end{figure}
%
%
%\begin{figure}[thpb]
%    \centering
%%    \psfrag{V0}[][]{\fontsize{8}{8}\selectfont\textsf{$V(0)$}}
%%    \psfrag{U1}[][]{\fontsize{8}{8}\selectfont\textsf{$U(t_1)$}}
%    %\psfrag{ep}[][]{\fontsize{8}{8}\selectfont\textsf{$EP$}}
%%    \psfrag{t}[][]{\fontsize{8}{8}\selectfont\textsf{$T$}}
%%    \psfrag{p}[][]{\fontsize{8}{8}\selectfont\textsf{$G(s)$}}
%%    \psfrag{k}[][]{\fontsize{8}{8}\selectfont\textsf{$K^{-1}$}}
%%    \psfrag{r}[][]{\fontsize{8}{8}\selectfont\textsf{$R$}}
%
%
%    \includegraphics[scale=0.3]{pumps_ss.eps}
%    \caption{Case 2 pumps control actions.}
%    \label{fig:pumps2}
%\end{figure}
%
%\begin{figure}[thpb]
%    \centering
%%    \psfrag{V0}[][]{\fontsize{8}{8}\selectfont\textsf{$V(0)$}}
%%    \psfrag{U1}[][]{\fontsize{8}{8}\selectfont\textsf{$U(t_1)$}}
%    %\psfrag{ep}[][]{\fontsize{8}{8}\selectfont\textsf{$EP$}}
%%    \psfrag{t}[][]{\fontsize{8}{8}\selectfont\textsf{$T$}}
%%    \psfrag{p}[][]{\fontsize{8}{8}\selectfont\textsf{$G(s)$}}
%%    \psfrag{k}[][]{\fontsize{8}{8}\selectfont\textsf{$K^{-1}$}}
%%    \psfrag{r}[][]{\fontsize{8}{8}\selectfont\textsf{$R$}}
%
%
%    \includegraphics[scale=0.3]{V_ss.eps}
%    \caption{Case 2 volume variation and capacity constraints.}
%    \label{fig:V2}
%\end{figure}
%
%

\begin{figure}[thpb]
    \centering
%    \psfrag{V0}[][]{\fontsize{8}{8}\selectfont\textsf{$V(0)$}}
%    \psfrag{U1}[][]{\fontsize{8}{8}\selectfont\textsf{$U(t_1)$}}
    %\psfrag{ep}[][]{\fontsize{8}{8}\selectfont\textsf{$EP$}}
%    \psfrag{t}[][]{\fontsize{8}{8}\selectfont\textsf{$T$}}
%    \psfrag{p}[][]{\fontsize{8}{8}\selectfont\textsf{$G(s)$}}
%    \psfrag{k}[][]{\fontsize{8}{8}\selectfont\textsf{$K^{-1}$}}
%    \psfrag{r}[][]{\fontsize{8}{8}\selectfont\textsf{$R$}}


    \includegraphics[scale=0.28]{Pump_total.eps}
    \caption{Case 2 pumps control actions.}
    \label{fig:pumps3}
\end{figure}

\begin{figure}[thpb]
    \centering
%    \psfrag{V0}[][]{\fontsize{8}{8}\selectfont\textsf{$V(0)$}}
%    \psfrag{U1}[][]{\fontsize{8}{8}\selectfont\textsf{$U(t_1)$}}
    %\psfrag{ep}[][]{\fontsize{8}{8}\selectfont\textsf{$EP$}}
%    \psfrag{t}[][]{\fontsize{8}{8}\selectfont\textsf{$T$}}
%    \psfrag{p}[][]{\fontsize{8}{8}\selectfont\textsf{$G(s)$}}
%    \psfrag{k}[][]{\fontsize{8}{8}\selectfont\textsf{$K^{-1}$}}
%    \psfrag{r}[][]{\fontsize{8}{8}\selectfont\textsf{$R$}}


    \includegraphics[scale=0.28]{Vol_total.eps}
    \caption{Case 2 volume variation and capacity constraints.}
    \label{fig:V3}
\end{figure}








\section{CONCLUSION}
In this article we have presented a new approach to the solution of
optimization problems with discrete control variables which yield
binary integer programs (BIP). The approach is to transform a single
BIP optimization by a LP and $N$ feasibility BIP with less number of
variables. In this way it is possible to solve optimization problems
with longer time intervals and with higher number of actuators,
because the BIP number of variables is reduced, improving the
algorithmic behavior in providing a solution. Moreover the
optimization problem is translated to a feasibility problem, in this
way only a feasible solution is needed. Finally the complexity of
the feasibility BIPs does not depend on the problem time horizon.

\bibliographystyle{IEEEtran}
\bibliography{oem_bip}






%   \begin{figure}[thpb]
%      \centering
%      %\includegraphics[scale=1.0]{figurefile}
%      \caption{Inductance of oscillation winding on amorphous
%       magnetic core versus DC bias magnetic field}
%      \label{figurelabel}
%   \end{figure}
%
%
%
%%%%%%%%%%%%%%%%%%%%%%%%%%%%%%%%%%%%%%%%%%%%%%%%%%%%%%%%%%%%%%%%%%%%%%%%%%%%%%%%%
%\section{ACKNOWLEDGMENTS}
%
%The authors gratefully acknowledge the contribution of National Research Organization and reviewers' comments.
%
%
%%%%%%%%%%%%%%%%%%%%%%%%%%%%%%%%%%%%%%%%%%%%%%%%%%%%%%%%%%%%%%%%%%%%%%%%%%%%%%%%%
%
%References are important to the reader; therefore, each citation must be complete and correct. If at all possible, references should be commonly available publications.
%
%\begin{thebibliography}{99}
%
%\bibitem{c1}
%J.G.F. Francis, The QR Transformation I, {\it Comput. J.}, vol. 4, 1961, pp 265-271.
%
%\bibitem{c2}
%H. Kwakernaak and R. Sivan, {\it Modern Signals and Systems}, Prentice Hall, Englewood Cliffs, NJ; 1991.
%
%\bibitem{c3}
%D. Boley and R. Maier, "A Parallel QR Algorithm for the Non-Symmetric Eigenvalue Algorithm", {\it in Third SIAM Conference on Applied Linear Algebra}, Madison, WI, 1988, pp. A20.
%
%\end{thebibliography}

\end{document}
